% THIS IS AN EXAMPLE DOCUMENT FOR VLDB 2012
% based on ACM SIGPROC-SP.TEX VERSION 2.7
% Modified by  Gerald Weber <gerald@cs.auckland.ac.nz>
% Removed the requirement to include *bbl file in here. (AhmetSacan, Sep2012)
% Fixed the equation on page 3 to prevent line overflow. (AhmetSacan, Sep2012)

\documentclass{vldb}
\usepackage{graphicx}
\usepackage{balance}  % for  \balance command ON LAST PAGE  (only there!)


\begin{document}

% ****************** TITLE ****************************************

\title{Client-Side Indexes for Fast Full-Text Searching}

% possible, but not really needed or used for PVLDB:
%\subtitle{[Extended Abstract]
%\titlenote{A full version of this paper is available as\textit{Author's Guide to Preparing ACM SIG Proceedings Using \LaTeX$2_\epsilon$\ and BibTeX} at \texttt{www.acm.org/eaddress.htm}}}

% ****************** AUTHORS **************************************

\numberofauthors{3}

\author{
\alignauthor
Amy X. Zhang\\
       \email{axz@mit.edu}
\alignauthor Lea Verou\\
       \email{leaverou@mit.edu}
\alignauthor Manali Naik\\
       \email{manalinaik@mit.edu}
}

\maketitle

\begin{abstract}

Many applications on the web use a combination of client-side and server-side data stores to facilitate fast interactive and data-intensive experiences. 
However, standard client side databases within browsers do not currently support full-text searching. 
In this paper, we describe a client-side search engine built on top of IndexedDB that makes use of several types of indexes common to many well-known server-side search engines.
We compare the performance of different indexes on different types of full-text content and queries and find that....
We also compare the performance of our system with that of fully server side systems and examine scenarios where a hybrid approach may be fastest. We find that...

\end{abstract}

\section{Introduction}

Today


\section{Background and Related Work}

There are a lot of cases where apps store a heavy amount of data in the browser. This study gives users ubiquitous access to data by allowing browser session migration \cite{lo2013imagen}

client-side profiles for personalized advertising, privacy concerns \cite{bilenko2011predictive}

Client-side database storage can improve the performance of data intensive websites by executing portions of web applications client-side and synchronizing with a web server.
\cite{benson2010sync}


Search engines can be useful in the absence of connectivity on mobile phones. This system builds on a user study showing that revisitation is common.
\cite{balasubramanian2012findall}

Previous research has demonstrated that it is feasible to store a reverse index within IndexedDB
\cite{lin:jscene} though it is much slower than using a server-side application such as Lucene.




\section{Lucy.js}



\section{Index Implementations}



\section{Evaluation}


\section{Discussion}

\section{Future Work}

\section{Conclusion}


\bibliographystyle{abbrv}
\bibliography{paper} 

\end{document}
